\chapter{Implementacja}
\section{Kompilacja projektu}
Kompilacja projektu składa się z 2 etapów – kompilacji programu uruchamianego w robocie oraz aplikacji mobilnej. W celu wgrania oprogramowania robota należy skompilować kod źródłowy w środowisku \textit{Eclipse}, a następnie utworzony plik binarny przesłać do procesora pojazdu przy użyciu środowiska \textit{STMStudio}. Rozwiązanie nie należy do najwygodniejszych, szczególnie na etapie testowania. Niestety jest to jedyny sposób, wynikający z błędów popełnionych na etapie projektowania, które zostaną opisane w podsumowaniu. Drugi etap jest mniej problematyczny, ponieważ nie wymaga integracji oprogramowania z niededykowanym urządzeniem.  Kompilacja kodu aplikacji mobilnej odbywa się w obrębie jednego narzędzia jakim jest \textit{Xcode}. 